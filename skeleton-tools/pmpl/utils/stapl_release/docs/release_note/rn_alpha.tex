\documentclass{report}
\begin{document}
\title{STAPL Alpha Release Notice}
\date{\today}
\maketitle

\newcommand{\stapl}{{\sc STAPL}}

% % % % % % % % % % % % % % % % % % % % % % % % % % % % % % % % % % % % %

\vspace*{8cm}
\begin{center}
Copyright (c) 2014, Texas Engineering Experiment Station (TEES), a
component of the Texas A\&M University System.

All rights reserved.
\end{center}
\pagebreak

% % % % % % % % % % % % % % % % % % % % % % % % % % % % % % % % % % % % %

\chapter{Introduction}

\section{Purpose}

These release notes contain the following:
\begin{itemize}
\item
A brief description of the features of the release,
\item
Instructions for installing,
\item
Description of those known high-impact defects that can be shown 
with a user source code example, and workarounds where available.
\end{itemize}

\subsection{Terminology}

\subsection{Alpha Release Documentation}

The following documentation is provided with the Alpha Release:

\begin{itemize}
\item
{\bf Reference Manual:} HTML automatically generated from \stapl\ source,
using the doxygen tool and comments embedded in the source.
To use this documentation, point your browser at /docs/html.
\item
{\bf Tutorial User Guide:} PDF which guides the new user through a set
of twenty graduated examples, showing how to perform common tasks.
\end{itemize}

% % % % % % % % % % % % % % % % % % % % % % % % % % % % % % % % % % % % %

\chapter{Software Features}

This chapter describes the features supported by the alpha release of \stapl.

Software features are organized in the following categories:

\begin{itemize}
\item
Container Support
\item
View Support
\item
Algorithm Support
\item
Nested parallelism
\item
Data redistribution
\item
Multi-protocol parallelism
\end{itemize}

% % % % % % % % % % % % % % % % % % % % % % % % % % % % % % % % % % % % %

\section{Container Support}

\noindent
These sequence containers are supported:

{\tt list, vector }

\noindent
These associative containers are supported:

{\tt set, map }

For the alpha release, the members of sets are limited  
boolean, character, integer, and floating point data types.

In a future release, \stapl\ will support containers as members of sets.

For the alpha release, the keys of a map are limited to 
boolean, character, integer, and floating point data types.

In a future release, \stapl\ will support containers as keys of a map.

There is no limitation on the data types of the values of a map.
These can be basic data types, or containers of any kind.

\noindent
These indexed containers are supported:

{\tt array, matrix, multiarray (vector is also indexed) }

\noindent
These relational containers are supported:

{\tt graph, dynamic\_graph }

\vspace{0.4cm}

\noindent
\textit{ The following associative containers are {\bf NOT} supported
in the alpha release.  }

{\tt multiset, multimap }

\noindent
\textit{ The following unordered associative containers are {\bf NOT} 
supported in the alpha release.  }

{\tt unordered set, unordered map }

% % % % % % % % % % % % % % % % % % % % % % % % % % % % % % % % % % % % %

\section{View Support}

\subsection{Abstract Data Types}

\noindent
These views model common abstract data types.
Most of them are associated with a container of the same name,
but their use is not limited to that associated container.
The following views in this category are supported:

{\tt array, array\_ro, vector, list, graph, map, set,
multi\_array }

\subsection{No Concrete Storage}

\noindent
These views generate values in a manner that does not
require concrete data storage associated with every element.
The following views in this category are supported:

{\tt overlap, repeated, counting }

\vspace{0.4cm}

\noindent
\textit{ The following views in this category are 
{\bf NOT} supported in the alpha release: }

{\tt functor }

\subsection{Modifying Views}

These views modify 
\begin{itemize}
\item
the values retrieved from the storage associated with a container, or
\item
the mapping function M or domain D, such that a different value is retrieved.
\end{itemize}
The following views in this category are supported:

{\tt transform, strided, filter, reverse }

\subsection{Sub-dividing Views}

\noindent
These views present elements that are subviews.

{\tt native, balance, explicit, partitioned }

\subsection{Compositions}

These views perform compositions on containers.

\noindent
The following views in this category are supported:

{\tt zip }

\vspace{0.4cm}

\noindent
\textit{ The following views in this category are {\bf NOT} supported 
in the alpha release:}

{\tt cross, union }

% % % % % % % % % % % % % % % % % % % % % % % % % % % % % % % % % % % % %

\section{Algorithm Support}

The following algorithms are supported:

\begin{itemize}
\item Non-modifying algorithms:

{\tt
for\_each, count, count\_if, min\_element, max\_element, 
find, find\_if, search\_n, search, find\_end, 
find\_first\_of, adjacent\_find, equal, mismatch, 
lexicographic\_compare
}

\item Modifying algorithms:

{\tt
for\_each, copy, copy\_backward, transform, merge, 
swap\_ranges, fill, fill\_n, generate, generate\_n, 
replace, replace\_if, replace\_copy, replace\_copy\_if
}

\item Removing algorithms:

{\tt
remove, remove\_if, remove\_copy, remove\_copy\_if, 
unique, unique\_copy
}

\item Mutating algorithms:

{\tt
reverse, reverse\_copy, rotate, rotate\_copy, 
next\_permutation, prev\_permutation, random\_shuffle
}

\item Sorting algorithms:

{\tt
sort, stable\_sort, partial\_sort, partial\_sort\_copy, 
nth\_element, partition, stable\_partition
}

\item Sorted range algorithms:

{\tt
binary\_search, includes, lower\_bound, upper\_bound, 
equal\_range, merge, inplace\_merge, adjacent\_find
}

\item Numeric algorithms:

{\tt
accumulate, inner\_product, adjacent\_difference, 
partial\_sum
}

\item C++ '11 STL algorithms:

{\tt
is\_partitioned, partition\_point, is\_sorted,
all\_of, any\_of, one\_of, none\_of,
is\_sorted\_until, is\_permutation,
copy\_n, partition\_copy
copy\_if, iota, minmax\_element,
find\_if\_not,
}

\item \stapl\ general algorithms not in STL:

{\tt
adjacent\_find, not\_equal,
n\_partition, sample\_sort, shuffle,
column\_sort, 
keep\_if, max\_value, min\_value,
}

\item \stapl\ graph algorithms:

{\tt
breadth\_first\_search, connected\_components, 
graph\_coloring, topological\_sort, hierarchical\_view, 
cut\_conductance, graph\_metrics, strongly\_connected\_components
}
\end{itemize}

% % % % % % % % % % % % % % % % % % % % % % % % % % % % % % % % % % % % %

\section{Nested Parallelism}

Nested parallelism supports the following features:

\begin{itemize}
\item
Uniform nested containers from the following:
vector, array, map, multiarray, graph, set
\item
Mixed nested containers from the following:
vector, array, map, multiarray, graph, set
\item
Data distribution for supported nested containers for all containers
at a specified nesting level, or for individual nested containers.
Distribution are specified when the container is constructed.
\end{itemize}

\noindent
\textit{ The following containers have specialized limitations when used in a nested fashion:}
\begin{itemize}
\item
set - can only be the innermost container (key semantics)
\item
map - nested container can only be the value, not the key (key semantics)
\end{itemize}

\noindent
\textit{ The following containers are {\bf NOT} supported for
nested parallelism in the alpha release: }

\noindent
matrix, list, 
unordered set, unordered map, multiset, multimap, multi\_graph

\noindent
\textit{ The following features are {\bf NOT} supported for
nested parallelism in the alpha release : }

\begin{itemize}
\item
Constructing non-innermost nested containers by insertion.
\item
Copying interior nested containers to or from the enclosing container.
\end{itemize}

% % % % % % % % % % % % % % % % % % % % % % % % % % % % % % % % % % % % %

\section{Data Redistribution}

The following features for data redistribution are supported:

\begin{itemize}
\item
For those containers for which it is possible, they have a "migrate" 
method, which moves an element specified by a GID to a designated
location.
\item
Each supported container has a "redistribute" method, 
which takes a view which specifies the new distribution.
\end{itemize}

% % % % % % % % % % % % % % % % % % % % % % % % % % % % % % % % % % % % %

\section{New Container Classes}

\noindent
The following non-STL containers are supported in the alpha release.

{\tt
matrix, multi-array, graph
}

% % % % % % % % % % % % % % % % % % % % % % % % % % % % % % % % % % % % %

\section{Multi-protocol Parallelism}

\noindent
Multi-protocol parallelism, in which the outer parallelism uses MPI,
and the inner parallelism uses OpenMP, is supported in the Alpha Release.

% % % % % % % % % % % % % % % % % % % % % % % % % % % % % % % % % % % % %

\chapter{Installation Procedures}

\section{Compiler Support}

\stapl\ supports the following compilers:

\begin{itemize}
\item
gcc versions 4.7.0 and higher.
\item
icpc (ICC) versions 13.1.2 20130514 and higher, using libstdc++ from
gcc 4.7.0.
\item
clang versions 3.3 and higher.
\end{itemize}

We plan to support the PGI C++ compiler in the \stapl\ beta release.

\section{Overview}

Source code for \stapl\ is delivered in the form of tar.gz archive,
plus scripts to configure, build and install \stapl.

The installation process is driven by shell scripts which do the following:

\begin{itemize}
\item
Configure \stapl\ for the version of Linux, BOOST location, install location
\item
Build libstapl.a and install
\item
Copy header files
\end{itemize}

\noindent
The scripts require that the installer have "sudo" privileges.

\section{Packaging Procedure}

The \stapl\ deliverables are packaged by the following process,
which is described here in order to enable understanding of
what is delivered with the \stapl\ distribution.  

\begin{itemize}
\item
Checkout source code from source control system.
\item
Strip source control information.
\item
Delete unnecessary files and directories: functionality tests, 
performance tests, software tools, internal documentation.
\item
Build the API doxygen-based HTML reference.
\item
Build LaTeX-based tutorial user guide.
\item
Create a gzip compressed tar file (defaults to "stapl.tar.gz")
\end{itemize}

\section{Installation Procedure}

\subsection{Extraction}

The first step in the installation process is to extract the sources
into a local file system.

\begin{verbatim}
% tar xvf stapl.tar.gz
\end{verbatim}

\subsection{Compilation}

The second step in installing \stapl\ is to compile the sources.
A script is provided which performs this task.

\begin{verbatim}
% cd stapl
% ./compile.sh
\end{verbatim}

This script has the following command line options:
\begin{itemize}
\item
\texttt{-openmp} : Compile mixed-mode support based on OpenMP
\item
\texttt{-cc=xx} : Change compiler to XX
\item
\texttt{-AR=XX} : Change archiver to XX
\end{itemize}

This script has the following features:

\begin{itemize}
\item
It automatically picks up Boost based on \texttt{BOOST\_ROOT}.
\item
It attempts to find the correct libstdc++ based on the GCC version.
\item
It creates a directory 
\texttt{build} and copies \texttt{tools} and \texttt{stapl} directories in it.
\item
It builds debug (\texttt{libstapl\_debug.a}) and release 
(\texttt{libstapl.a}) versions.
\item
It builds OpenMP versions if requested 
(\texttt{libstapl\_omp\_debug.a / libstapl\_omp.a}).
\item
It puts libraries in \texttt{build/lib}.
\item
It puts environment variable scripts in \texttt{build/bin}.
\end{itemize}

\subsection{Installation}

The third step in installing \stapl\ is to install the libraries
and header files.  A script is provided to perform this task.

\begin{verbatim}
% sudo install.sh PATH
\end{verbatim}

This script has the following features:

\begin{itemize}
\item
It is a tool for installing \stapl\ into a directory.
\item
It copies the \texttt{build} directory to the location specified by the
\texttt{PATH} command line argument.
\item
It is typically invoked with "sudo".  Because it modifies system directories,
it cannot be run by an ordinary user account without root privileges.
\end{itemize}

\subsection{Configuration}

The final step in installing \stapl\ is to configure the user's environment
to use \stapl.  Configuration is done on a per-user basis.
Scripts are provided to perform this task.
	
\begin{verbatim}
$ source staplvars.sh
\end{verbatim}

\texttt{staplvars.sh [debug]} and \texttt{staplvars\_omp.sh [debug]}

These scripts have the following features:
\begin{itemize}
\item
They set up the environment for compiling \stapl\ applications.
\item
They typically used with \texttt{source}, in order to set the shell variables.
\item
If they are passed debug as argument, 
then they bring in the variables for the debug versions.
\end{itemize}

These scripts set the following variables:
\begin{itemize}
\item
\texttt{STAPL\_ROOT} : \stapl\ installation path
\item
\texttt{STAPL\_FLAGS} : \stapl\ required flags
\item
\texttt{STAPL\_LIBS} : \stapl\ required libraries
\end{itemize}

Given appropriate values for these variables, 
the user could compile a \stapl\ application as follows
(where \texttt{CC} has been assigned the path to the correct compiler):

\begin{verbatim}
$CC $STAPL_FLAGS $CXXFLAGS myapp.cpp -o myapp $LIBS $STAPL_LIBS
\end{verbatim}

Given this compilation, the user could invoke the resulting
application with a command line such as this
(where \texttt{LAUNCH} has been assigned the path to the appropriate launcher, 
such as mpiexec, aprun, etc.):

\begin{verbatim}
$ $LAUNCH -n 42 -N 8 -d 6 ./myapp
\end{verbatim}

% % % % % % % % % % % % % % % % % % % % % % % % % % % % % % % % % % % % %

\chapter{Known Defects and Workarounds}

\end{document}
