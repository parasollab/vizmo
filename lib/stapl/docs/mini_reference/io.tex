\chapter{Parallel Input/Output}

\section{User Messages}

STAPL is a SPMD system.  So, every statement in a program
run using STAPL will be executed by every process.  This means typical
C++ input and output statements will perform in duplicate, unless something 
is done to avoid this.

The STAPL basic parallel algorithms ( \texttt{
map\_func, map\_reduce, scan, reduce, serial, serial\_io, do\_once ) }
distribute execution among processing locations, rather than duplicate it.
The last two constructs are used to perform I/O without duplication.

The \texttt{ do\_once }
is used to generate messages to the user.  It executes a work function one
time on location zero.  Define an appropriate work function as follows:

\begin{verbatim}
#include <iostream> 
#include <stapl/utility/do_once.hpp>
#include <stapl/runtime.hpp> 

template<typename Value>
struct msg_val {
private:
  const char *m_txt;
  Value m_val;
public:
  msg_val(const char *text, Value val)
    : m_txt(text), m_val(val)
  { }

  typedef void result_type;
  result_type operator() () {
    cout << m_txt << " " << m_val << endl;
  }
  void define_type(stapl::typer& t) {
    t.member(m_txt);
    t.member(m_val);
  }
};
\end{verbatim}

Execute the work function in your application as follows:

\begin{verbatim}
  stapl::do_once( msg_val<int>( "The answer is ", 42 ) );
\end{verbatim}

If you are familiar with the C++11 \texttt{lambda} feature, you can use it
with \texttt{do\_once} to generate user messages, without having to use
a separate work function.

\begin{verbatim}
  stapl::do_once([&](void) {
    cout << "The answer is " << 42 << endl;
  });
\end{verbatim}

% % % % % % % % % % % % % % % % % % % % % % % % % % % % % % % % % % % % % % % 

\section{Serial Data Values}

The 
\texttt{
serial\_io }
basic algorithm is used to apply a work function to the elements of
the input views in serial order.  The work function will only be executed
on location 0.

The STAPL \texttt{stream} class is used to encapsulate the C++ 
\texttt{iostream} file classes in a form that can be used with STAPL.

In order to read or write STAPL containers using \texttt{serial\_io},
we need a work function like the following.

\begin{verbatim}
class get_val_wf
{
private:
  stapl::stream<ifstream> m_zin;
public:
  get_val_wf(stapl::stream<ifstream> const& zin)
    : m_zin(zin)
  { }
  typedef void result_type;
  template <typename Ref>
  result_type operator()(Ref val) {
    m_zin >> val;
  }
  void define_type(stapl::typer& t) {
    t.member(m_zin);
  }
};
\end{verbatim}

The work function is used as follows.  First, we open the file, and then
we apply 

\begin{verbatim}
  stapl::vector<int> a_ct(size);
  stapl::vector_view<vector<int> > a_vw(a_ct);
  const char *file_name = "path";
  stapl::stream<ifstream> zin;
  zin.open(file_name);
  stapl::serial_io(get_val_wf(zin), a_vw);
\end{verbatim}

Performing output from a STAPL parallel container is done in a completely
analogous manner, substituting \texttt{ofstream} for \texttt{ifstream}, and
the insert operator $<<$ for the extract operator $>>$.

% % % % % % % % % % % % % % % % % % % % % % % % % % % % % % % % % % % % % % % 

\section{Parallel Data Values}

The STAPL graph class provides some special methods that make it possible
to perform parallel I/O on graphs stored in files.  One method can perform
\textit{sharding} on an input file.  This process splits a file into a set
of files, with a metadata file that describes the set as a whole.  These
files can then each be read independently by separate processing locations.
The sharder is called as follows;

\begin{verbatim}
  const char *file_name = "path";
  int block_size = 100;
  stapl::graph_sharder(file_name,block_size);
  stapl::rmi_fence();
\end{verbatim}

Sharding is possible for graphs because the edge list representation
that is stored on a file can be easily be made into self-describing parts.

A file containing a graph edge list that has been sharded can be read
in parallel with code that looks like the following

\begin{verbatim}
typedef stapl::graph<stapl::UNDIRECTED, stapl::NONMULTIEDGES, int, int>
        graf_int_tp;
typedef stapl::graph_view<graf_int_tp> graf_int_vw_tp;

graf_int_vw_tp gr_vw = stapl::sharded_graph_reader<graf_int_tp>
                       (filename, stapl::read_adj_list_line() );
\end{verbatim}

Future releases of STAPL may included sharding for representations of
other data structures which can be made self-identifying, such as sparse arrays.
